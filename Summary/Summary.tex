{\let\clearpage\relax \chapter*{Executive Summary}}\label{chap:Summary}
\noindent The Einstein Telescope (ET) is a planned European 3rd Generation
Gravitational Wave (GW) Observatory, a new research infrastructure designed to observe the entire Universe using gravitational waves. ET will be a multi-interferometer observatory covering the whole gravitational wave spectrum observable from Earth using interferometric detectors. 

\vspace{3mm}
This document provides a summary of the science case for ET, the design and feasibility of the required infrastructure, and the design study of the ET detectors. For a more extensive treatment of the same topics please see the ET Design Report Update 2020, available at: \url{https://tds.virgo-gw.eu/ql/?c=15418}.
%TODO add reference to the larger design document

\vspace{3mm}
%\section*{Introduction}
After the 11 detections (10\, BBH,  and a BNS seen in coincidence with a GRB) observed in the first two joint observation runs (O1 and O2) of the LIGO and Virgo detectors, GW candidates have been detected during O3 at an average rate of approximately one per week, and public alerts have been released for potential follow-up investigations with other telescopes. From O3 LIGO and Virgo have already reported another BNS observation, and interesting physical effects associated with two other BBH events. Additional important O3 detections are still to be announced.
For the next decade, the scientific program of the network of the LIGO, Virgo and KAGRA detectors foresees two more data taking periods (O4 in [2022/2023] and O5 in [2025/2026]),
alternating with periods of detector upgrades, until $\sim$ 2026~-~2027. During
this period, the detector sensitivity will increase by up to a factor $\sim 3-5$
for Virgo and $\sim 3$ for LIGO, i.e. about a factor of two
beyond their design sensitivities. The Japanese detector KAGRA plans to join the network
in 2020 and LIGO India will be operational around 2025. Concrete plans for the
period after 2027 have not yet been made. However, it is likely that the three
LIGO detectors together with Virgo and KAGRA (five detectors in total) will be operating until the end of the next decade. 

The current infrastructures will then have reached physical limitations,
both in terms of durability and performance, and hence won't allow further 
significant sensitivity improvements. A new, third generation of
Earth-based detectors will be needed.
The Virgo and LIGO infrastructures initially carried the first generation
(1G) of GW detectors, and now host the upgraded second-generation (2G) detectors
Advanced Virgo and Advanced LIGO.
The Einstein Telescope will be a new infrastructure, which will first host a 3G
observatory. A similar effort is being pursued in the US (Cosmic Explorer).
After a design and preparatory phase, the Einstein Telescope can be constructed
between 2026 and 2035 and then remain in operation for a period of 50 years. 

\vspace{5mm}
\section*{The instrument}
ET will improve the sensitivity by an order of magnitude with respect to the
design sensitivity of Advanced Virgo and Advanced LIGO and furthermore extend
the observation band towards lower frequencies, i.e. down to 
about 3\,Hz
(compared to $\sim 10$\,Hz for Virgo). ET will be based on
the same basic concept demonstrated in the framework of LIGO and Virgo: a
modified Michelson interferometer, with Fabry-Perot cavities in the arms and the
techniques of power recycling and signal recycling. However, ET's ambitious
sensitivity target, in particular at low frequencies, is based on several 
technology innovations. Specific ET design concepts are: 

%\vspace{1mm}
%\noindent 
\textit{Triangle}\quad ET will be composed of three nested detectors in a
configuration of an equilateral triangle, pair-wise sharing a (10\,km) tunnel.
This configuration will enable ET to resolve the GW polarization and provide
continuous operation during maintenance with a minimum physical infrastructure.

%\vspace{1mm}
%\noindent 
\textit{10\,km}\quad The ET detectors will have 10\,km long arms, to
increase the signal produced by the GWs. 
%and thereby to reduce the impact of most of the sensitivity-limiting noises. 
This change will provide a factor of three improvement compared to Virgo (3\,km long) with respect to virtually all of the sensitivity-limiting noises.

%\vspace{1mm}
%\noindent  
\textit{Xylophone}\quad 
Each of the three ET detectors will be composed of a pair of complementary interferometers, one with a peak sensitivity at low frequencies and the other with a sensitivity optimised for higher frequencies. The reason is to separate the challenges related to the use of cryogenic techniques (needed to reduce the Brownian thermal noise of suspensions and mirrors) together with high power stored in the arms (needed to reduce the photon shot noise). The 
%cryogenic and 
low-power (cold) detector, operating at a temperature of 20\,K,  will be optimised for low frequency gravitational wave sources (ET-LF) and the high-power (warm) detector, operating at room temperature, will work at high frequencies (ET-HF). 

\textit{Underground operation}\quad In order to reduce the impact of seismic noise and gravity gradient noise induced by seismic waves and 
%motion of air masses
compression waves of the surrounding air, ET will be built underground. The underground operation will allow to extend the frequency band of the oservatory down to a few Hz. 
The three 10\,km long tunnels, each having an inner diameter of 6.5\,m and containing 4 vacuum pipes, and the caverns containing the large vacuum tanks, will be excavated using well-established tunneling and underground excavation techniques. 
Besides the main caverns (at the vertices of the triangle), several auxiliary caverns will host further interferometer components. 

\section*{The site}
One of the consequences of the extension of the observation band towards lower frequencies is that environmental disturbances and therefore the quality of the observatory location play an increasingly important role. A strong reduction of environmental noise is achieved by placing the detectors in an underground location, provided that a suitable site is chosen.
The key evaluation criteria for the site selection for the Einstein Telescope include: impact on infrastructure lifetime, observatory sensitivity, observatory operation and duty cycle, site-quality preservation, construction cost, and socio-economic impact of the observatory. Two candidate sites have been identified for a detailed site-characterization:
one in the north of Lula on Sardinia, and one in the Meuse-Rhine Euroregion. 

\section*{The technologies}

The ET design is based on well proven and experimentally tested concepts. However, to achieve the ambitious sensitivity goal of the Einstein Telescope
it will be necessary to exploit many state-of-the-art technologies and drive
them to their physical limits. The Einstein Telescope combines the %well-proven
technologies from the Advanced LIGO and Virgo detectors (ultra-sensitive optical interferometry, complex active and passive seismic isolation systems and injection of non-classical 'squeezed' light)  with upgrades envisaged for the Advanced Detectors (cryogenic mirrors and frequency-dependent squeezing). 
The infrastructure is designed to accommodate several technology upgrades during 50\,years of operation.

Cryogenic optics will be tested in specific testbenches and integrated prototypes within the ET consortium. In addition, important experience will be gained from the operation of the the Japanese project KAGRA, which already uses cryotechnology. Frequency dependent squeezing will be tested in upgraded versions of Virgo and LIGO.  Specific technologies implemented in ET are the
following: 

%\vspace{1mm}
%\noindent
%\textit{Light source and squeezing}

%\vspace{1mm}
%\noindent
\textit{Seismic isolation and Suspensions}\quad Seismic isolation and suspension
systems are needed to isolate the mirrors from the seismic motion of the ground.
ET is aiming at a lower cut-off frequency than the 2nd detector generation. The
baseline for ET, defined in the 2011 Conceptual Design Report and detailed
there, consists of using a longer Virgo-style Superattenuator. An increased
length (17\,m) allows to reduce the normal mode frequencies and push the `seismic
wall' down to $\sim 2$\,Hz. A possible alternative to be investigated via a dedicated research and development program could be coupling a Superattenuator to an inertial platform actively controlled in six degrees of freedom.

%\vspace{1mm}
%\noindent
\textit{Materials for mirrors and suspensions}\quad
Monolithic suspensions, i.e. from the same material as the mirrors, will be used in the cryogenic and in the room-temperature interferometers of ET for managing suspension thermal noise.
%, which is most important in the mid-frequency band. 
The HF interferometer will use special-grade fused silica mirrors and suspensions, as used in GEO\,600, Advanced Virgo and Advanced LIGO. The ET-LF interferometer will use ultra-pure crystalline silicon. The KAGRA detector currently uses Sapphire as a cryogenic material, providing a valuable in-situ test of an alternative material.

%\vspace{1mm}
%\noindent
\textit{Vacuum}\quad
To keep the residual refractive index fluctuations low enough, the 10\,km optical path between the mirror test masses must be evacuated. The residual gas composition will be dominated by hydrogen in the presence of water and other gases; we will keep the total residual pressure at about $10^{-10}$\,mbar, which corresponds to a strain noise level below $10^{-25}\, \mathrm{Hz}^{-1/2}$. The hydrocarbon partial pressure ($>100~amu$) will be below a level of $10^{-14}$\,mbar.

%\vspace{1mm}
%\noindent
\textit{Cryogenics}\quad
The ET-LF test masses will be cooled to a temperature below 20\,K in order to
reduce thermal noise of the interferometers. 
The last stage of the suspension is composed by the test mass, suspended from a
penultimate mass (the so called "marionette"), a mechanical system able to
steer the test mass in several degrees of freedom. 
The silicon mirror, i.e. the main test mass,  is suspended from this penultimate
mass by silicon fibres. The total mass of this stage will be about one ton. It will be cooled in a dedicated cryostat, which will include two radiation shields at 8\,K and 80\,K. Removal of heat from absorbed laser light or environmental thermal radiation is done %almost entirely 
by heat conduction through the suspension fibres.

%\vspace{1mm}
%\noindent
\textit{Computing}\quad
The data collected by ET must be analyzed in real time to generate timely alerts for multi-messenger observations together with other astronomical observatories, such as optical telescopes. 
The deep analyses of individual GW observations requires an enormous amount of computing power due to the sheer number of events, the lower cut-off frequency of ET compared to the current detector generation, and the expected better signal quality and therefore a larger number of necessary templates. Since this computing capacity must be available 24/7, dedicated data centers will be necessary. However, a mere increase in computing capacity will not be sufficient. In addition, new, more efficient algorithms will have to be developed to meet the increased demand. 

\section*{The science}
\label{sect:ScienceCaseKeyQuestions}

%We conclude with a summary of the key scientific questions that ET will be able to tackle. 

GW detection has literally opened a new window on the Universe. With new
third-generation observatories such as ET we will begin to look far out through
this window. As with any scientific enterprise of this scale, there will be
certain questions for which, based on our current understanding, we can say that
ET is guaranteed to provide the answers, but ET will also be a discovery
machine. It will venture into unexplored territories where further surprises are
expected. A summary of the key science capabilities is as follows:

%\vspace{1mm}
%\noindent
(1)  ET will \textbf{detect binary black hole coalescences up to cosmological distances}.
For a total mass of the system between a few tens and a few hundreds solar
masses, as typical of the population of BH binaries revealed by 2G detectors, ET
will be able to detect their coalescence up to redshift $z\sim 20$ and higher,
see Fig.~\ref{fig:gw_horizons} on page~\pageref{fig:gw_horizons}. The corresponding rates will be of order
$10^5-10^6$ events per year. This  will  provide a census of the population of
BHs across the whole epoch of star formation and beyond, answering crucial
questions on the progenitors, formation, binary evolution and demographics of
stellar BHs. The astrophysical potential in this direction is
guaranteed. An observatory network of two or more 3rd-generation observatories 
would of course be beneficial, in particular for
source localization, but even ET as a single observatory 
is adequate to uncover much of this compelling science.

%\vspace{1mm}
%\noindent
(2) ET will \textbf{extend the region of BH masses} compared to that explored by 2G detectors, including sources of several hundreds of solar masses, that could be detected up to redshifts of order 10 or more, see Fig.~\ref{fig:gw_horizons}, 
and sources of several thousands solar masses, that could be detected up to $z\sim 1-5$.  
This opens the possibility of detecting  these intermediate mass BHs, providing
the first clear evidence for their existence and  studying the possibility that
they are the seeds of the supermassive BHs in the center of galaxies. On the
low-mass side, ET would detect, up to $z\sim 0.5-1$,  the coalescence of
hypothetical binary BHs with a total mass of order one solar mass; any BH with
such a mass would necessarily be of primordial, rather than stellar, origin.

%\vspace{1mm}
%\noindent
(3) ET will \textbf{detect the coalescence of binary neutron stars up to} $\mathbf{z\simeq\,2-3}$, with a rate of about $6\times 10^4$ events per year. This range  reaches the peak of the star formation rate and therefore covers the vast majority of NS binaries coalescing throughout the Universe. This will allow us to investigate their formation mechanisms, evolution, and demographics. The sensitivity of ET in the high-frequency regime will allow us to access the GW signal of the merger phase that is inaccessible to 2G detectors and carries detailed information on the internal structure of neutron stars and on their equation of state. This will have  important implications also for fundamental physics, allowing us to study QCD at ultra-high density and  the possibility of phase transitions in the NS core, such as a transition to deconfined quarks or the formation of exotic states of matter. These detections, and a rich science output coming from them, are guaranteed. Again, these goals can be obtained even by ET as a single observatory. A network of three 3G observatories would bring, on top of this, the possibility of accurate localization of the source, allowing to give information to electromagnetic telescopes necessary to identify an electromagnetic counterpart and perform multi-messenger studies.

%\vspace{1mm}
%\noindent
(4) ET  could \textbf{detect several new astrophysical sources of GWs}, such as signals
emitted during core collapse supernovae, continuous signals from isolated rotating NSs, and possibly burst signals from NSs. While not guaranteed, these signals would bring rich information. Detecting the GWs from supernovae would elucidate the mechanisms of supernova explosions and its post-collapse phase.
The detection of continuous GWs from NSs would allow us to explore the condition of formation and evolutions of isolated NS, providing information on their spin, thermal evolution and magnetic field. ET will be able to detect `mountains' on the surface of a NS as small as $10^{-2}$~mm, which in turn would again give us information on the inner structure of NS and on the corresponding aspects of nuclear and particle physics, such as the existence of exotic matter in the NS core.

%\vspace{1mm}
%\noindent
(5) The waveform from the loudest BH-BH and NS-NS coalescences will be \textbf{observed
by ET with exquisite precision}. This will allow accurate tests of General
Relativity, both in the inspiral phase, where one can test the validity of the
post-Newtonian expansion of GR to sub-permille accuracy, and in the merger and
post-merger phase. The latter is particularly interesting since it would allow
us to test the nature of BHs and the dynamics of space-time close to the horizon
of the final BH, through the observation of  the frequencies and lifetimes of
its longer-lived quasi-normal modes. This would allow us to perform 
%for the first time 
new accurate quantitative tests of the predictions of GR in this extreme
domain. The possibility of performing such accurate tests is guaranteed, and can
also  be performed by ET as a single observatory. These tests could also in
principle lead to  surprises, such as revealing  the existence of exotic compact
objects, and could even carry observable imprints of quantum gravity effects.
While the latter goals are more speculative, their impact would be
revolutionary.

%\vspace{1mm}
%\noindent
(6) ET will \textbf{test several dark matter candidates}, such as primordial black
holes, ultralight scalars or vector fields, or dark matter particles accreting on compact objects.
ET will be able to explore these possibilities even as a single observatory.  

%\vspace{1mm}
%\noindent
(7) ET will \textbf{explore the nature of dark energy} and the possibility of
modifications of General Relativity at cosmological distances. The crucial point
here is again the ability to detect compact binary coalescences up to
cosmological distances, providing an absolute measurement of their distance. The
relation between luminosity distance and redshift, in the range of redshifts
explored by  ET, carries very distinctive signature of the dark energy sector of
a modified gravity theory, through the dark energy equation of state and,
especially, through  an observable related to modified GW propagation. The
latter is a particularly powerful probe of dark energy, which is accessible only
by GW observations.  From the point of view of cosmology, ET is guaranteed to
obtain  important results (accurate measurement of $H_0$, significant limits on
the equation of state of dark energy), complementary to measurements obtained
with electromagnetic probes. 
The possibility of detecting  modifications of
General Relativity at cosmological scales and understanding the origin of dark
energy is not guaranteed, but would be revolutionary.

%\vspace{1mm}
%\noindent
(8) ET will \textbf{search for stochastic backgrounds of GWs}, which are relics of the
earliest cosmological epochs.  Such a background, if detected, would carry
information of the earliest moment of the Universe (much earlier than from $\mu$-wave observations), and on physics at the
corresponding high-energy scales, that is inaccessible  by electromagnetic
(or neutrino) observations or with particle accelerators. Stochastic
backgrounds of cosmological origin in the ET frequency window and sensitivity depend on physics beyond the Standard Model. Thus, the predictions are unavoidably uncertain, and
the gain from a successful detection would be correspondingly high, allowing us
to explore the earliest moments after the Big Bang.

